\documentclass[14pt]{beamer}
\usepackage[T2A]{fontenc}
\usepackage[utf8]{inputenc}
\usepackage[english,russian]{babel}
\usepackage{amssymb,amsfonts,amsmath,mathtext}
\usepackage{cite,enumerate,float,indentfirst}
\usepackage[export]{adjustbox}
\usepackage{wrapfig}
\usepackage{pgfplots}
\pgfplotsset{compat=1.9}

\graphicspath{{images/}}

% The Beamer class comes with a number of default slide themes
% which change the colors and layouts of slides. Below this is a list
% of all the themes, uncomment each in turn to see what they look like.

%\usetheme{default}
%\usetheme{AnnArbor}
%\usetheme{Antibes}
%\usetheme{Bergen}
%\usetheme{Berkeley}
%\usetheme{Berlin}
%\usetheme{Boadilla}
\usetheme{CambridgeUS}
%\usetheme{Copenhagen}
%\usetheme{Darmstadt}
%\usetheme{Dresden}
%\usetheme{Frankfurt}
%\usetheme{Goettingen}
%\usetheme{Hannover}
%\usetheme{Ilmenau}
%\usetheme{JuanLesPins}
%\usetheme{Luebeck}
%\usetheme{Madrid}
%\usetheme{Malmoe}
%\usetheme{Marburg}
%\usetheme{Montpellier}
%\usetheme{PaloAlto}
%\usetheme{Pittsburgh}
%\usetheme{Rochester}
%\usetheme{Singapore}
%\usetheme{Szeged}
%\usetheme{Warsaw}

% As well as themes, the Beamer class has a number of color themes
% for any slide theme. Uncomment each of these in turn to see how it
% changes the colors of your current slide theme.

%\usecolortheme{albatross}
%\usecolortheme{beaver}
%\usecolortheme{beetle}
%\usecolortheme{crane}
%\usecolortheme{dolphin}
%\usecolortheme{dove}
%\usecolortheme{fly}
%\usecolortheme{lily}
%\usecolortheme{orchid}
%\usecolortheme{rose}
%\usecolortheme{seagull}
%\usecolortheme{seahorse}
%\usecolortheme{whale}
%\usecolortheme{wolverine}

%\setbeamertemplate{footline} % To remove the footer line in all slides uncomment this line
%\setbeamertemplate{footline}[page number] % To replace the footer line in all slides with a simple slide count uncomment this line

\setbeamertemplate{navigation symbols}{} % To remove the navigation symbols from the bottom of all slides uncomment this line

\setbeamercolor{footline}{fg=red}
\setbeamertemplate{footline}{
  \leavevmode%
  \hbox{%
  \begin{beamercolorbox}[wd=.333333\paperwidth,ht=2.25ex,dp=1ex,center]{}%
    М. И. Глухова (Мехмат ЮФУ)
  \end{beamercolorbox}%
  \begin{beamercolorbox}[wd=.333333\paperwidth,ht=2.25ex,dp=1ex,center]{}%
    Ростов-на-Дону, 2015
  \end{beamercolorbox}%
  \begin{beamercolorbox}[wd=.333333\paperwidth,ht=2.25ex,dp=1ex,right]{}%
  Стр. \insertframenumber{} из \inserttotalframenumber \hspace*{2ex}
  \end{beamercolorbox}}%
  \vskip0pt%
}

\newcommand{\itemi}{\item[\checkmark]}

\title{\small{Программная реализация алгоритмов поиска \\аналитических кривых на изображении}}
\vspace{15pt}%
\author{\small{%
М. И. Глухова\\%
\emph{Направление подготовки:}~Фундаментальная информатика \\и информационные технологии\\%
\emph{Руководитель:}~проф., д.ф.-м.н. В. С. Пилиди}\\%
\vspace{15pt}%
    Южный федеральный университет\\
	Институт математики, механики и компьютерных наук
    имени~И.\,И.\,Воровича%
}
\date{\small{Ростов-на-Дону, 2015}}

\begin{document}

%------------------------------------------------
\maketitle
%------------------------------------------------

\begin{frame}
\frametitle{Содержание} % Table of contents slide, comment this block out to remove it
\tableofcontents % Throughout your presentation, if you choose to use \section{} and \subsection{} commands, these will automatically be printed on this slide as an overview of your presentation
\end{frame}

%------------------------------------------------
\section{Постановка задачи}
%------------------------------------------------

\begin{frame}
\frametitle{Постановка задачи}
\begin{block}{}
\begin{itemize}
  \item \textbf{Разработать алгоритм для поиска эллипсов на изображении.} 
  \item \textbf{Реализовать разработанный алгоритм.} 
  \item \textbf{Протестировать работу полученной программы.} 
\end{itemize}
\end{block}
\end{frame}

%------------------------------------------------
\section{Введение}
%------------------------------------------------
%------------------------------------------------
\begin{frame}
\frametitle{Параметры эллипса}
\small
\begin{block}{Каноническое уравнение}
$$ \frac{x^2}{a^2} + \frac{y^2}{b^2} = 1 $$
\end{block}

\begin{block}{Поворот}
$$ \frac{(x\cos A + y\sin A)^2}{a^2} + \frac{(x\sin A - y\cos A)^2}{b^2} = 1 $$
\end{block}

\begin{block}{Смещение}
  $$ \frac{((x - x_0)\cos A + (y - y_0)\sin A)^2}{a^2}  + $$
  
  $$ \frac{((x - x_0)\sin A - (y - y_0)\cos A)^2}{b^2} = 1 $$
\end{block}
\normalsize
\end{frame}
%------------------------------------------------

\section{Преобразование Хафа}
\begin{frame}
\frametitle{Преобразование Хафа}
\begin{block}{Общая схема}
\begin{itemize}
  \item Определение параметров
  \item Голосование
  \item Выбор кандидатов   
\end{itemize}
\end{block}

\begin{block}{Сложность}
\(O(A^{m-2})\), где \\\(A\) --- количество точек границ, \\\(m\) --- количество параметров.
\end{block}
\end{frame}

%------------------------------------------------
\section{Преобразование Хафа для поиска эллипсов}
%------------------------------------------------

%------------------------------------------------
\begin{frame}
\frametitle{Алгоритм поиска эллипсов}
\begin{block}{Этапы алгоритма}
\begin{enumerate}
  \item Предобработка
  \item Построение иерархической пирамиды
  \item Поиск на изображении наименьшего размера
  \item Итеративное уточнение параметров
\end{enumerate}
\end{block}
\end{frame}

%------------------------------------------------
\subsection{Предобработка}
%------------------------------------------------
\begin{frame}
\frametitle{Предобработка}
\begin{block}{}
\begin{itemize}
  \item Преобразование к размеру $N \times N$, где $N = 2^t$
  \item Снижение уровня шума
  \item Выделение границ
\end{itemize}
\end{block}
\end{frame}

%------------------------------------------------
\subsection{Построение иерархической пирамиды}
%------------------------------------------------
\begin{frame}
\frametitle{Построение иерархической пирамиды}
\begin{block}{}
\begin{itemize}
  \item Первый уровень: 1, если является пикселом границы, 0 - не является
  \item Следующие уровни строятся суммированием соответствующих значений на предыдущем
\end{itemize}
\end{block}
\begin{figure}[H]
  \center
  \includegraphics[width=0.7\linewidth]{pyramidcreation}
\end{figure}
\begin{block}{Теоретическая оценка сложности}
\(\Theta(N)\), где $N$ --- размер изображения.
\end{block}
\end{frame}

%------------------------------------------------
\subsection{Первичный поиск}
%------------------------------------------------
\begin{frame}
\frametitle{Первичный поиск}
Проводится на уменьшенном изображении.
\begin{block}{}
\begin{enumerate}
  \item Выбор двух краевых точек
  \item Определение центра, большой полуоси
  \item Определение малой полуоси голосованием
\end{enumerate}
\end{block}
\begin{block}{Теоретическая оценка сложности}
$$O(k_0^3)$$ где $k_0$ --- количество значимых пикселов на уменьшенном изображении.
\end{block}
\end{frame}

%------------------------------------------------
\begin{frame}
\frametitle{Первичный поиск}
Пусть зафиксированы точки $(x_1, y_1)$ и $(x_2, y_2)$.
  \center
\includegraphics[width=0.40\linewidth]{ellipse1}
$$b = ?$$
\end{frame}

%------------------------------------------------
\begin{frame}
\frametitle{Определение малой полуоси}
\begin{block}{}
\begin{enumerate}
  \item Для каждой точки границы $(x, y)$ предполагаем, что она лежит на границе эллипса
  \item Вычисляем $b$
  \item Увеличиваем значение аккумуляторного массива для найденного $b$
  \item Выбираем максимальное значение из аккумуляторного массива
\end{enumerate}
\end{block}
\end{frame}

%------------------------------------------------
\subsection{Уточнение параметров}
%------------------------------------------------
\begin{frame}
\frametitle{Уточнение параметров}
\begin{block}{}
\begin{enumerate}
  \item Если на предыдущем этапе найден эллипс с краевой точкой \((x_1, y_1)\), то на новом этапе поиск проводится только среди точек \((2x_1 \pm 1, 2y_1 \pm 1)\).
  \item Аналогично для \((x_2, y_2)\).
\end{enumerate}
\end{block}
Уточнение проводится итеративно, пока не достигнут исходный размер.
\begin{block}{Теоретическая оценка сложности}
$$O(N\log(N))$$ где $N$ --- размер изображения.
\end{block}
\end{frame}

%------------------------------------------------

\begin{frame}
\frametitle{Теоретическая оценка сложности алгоритма}
\begin{block}{Оценка по времени}
$$O(k_0^3) + O(N\log(N)) + \Theta(N)$$ где $k_0$ --- количество значимых пикселов на уменьшенном изображении, $N$ --- размер изображения.
\end{block}
\begin{block}{Оценка по памяти}
$$O(N^2)$$ где $N$ --- размер изображения.
\end{block}
\end{frame}

%------------------------------------------------
\section{Реализация}
%------------------------------------------------
\begin{frame}
\frametitle{Реализация}
\begin{block}{Использованные инструменты}
\begin{itemize}
  \item Язык реализации: C++ (g++ версии 4.9.2)
  \item OpenCV 3.0
\end{itemize}
\end{block}
\end{frame}

%------------------------------------------------

\begin{frame}
\frametitle{Результаты работы программы}
\begin{wrapfigure}{l}{0.32\linewidth} 
\vspace{-3ex}
\includegraphics[width=0.32\textwidth,left]{res}
\label{fig:res}
\end{wrapfigure}
Найденные эллипсы:
\small
\begin{itemize}
  \item $(x_0 = 210, y_0 = 149, a = 133.795, b = 65, A = -0.343024)$
  \item $(x_0 = 319, y_0 = 94, a = 135.533, b = 68, A = -0.0221366)$
  \item $(x_0 = 244, y_0 = 319, a = 111.005, b = 64, A = -0.00900877)$
\end{itemize}
\normalsize
\end{frame}

%------------------------------------------------
\begin{frame}
\frametitle{Результаты работы программы}
\center
\small
\begin{tabular}{ cc }
  \fbox{\includegraphics[width=0.25\textwidth]{r1}} & \fbox{\includegraphics[width=0.25\textwidth]{r2}} \\
  $[512\times512]$ & $[1024\times1024]$\\
  \fbox{\includegraphics[width=0.25\textwidth]{res6}} & \fbox{\includegraphics[width=0.25\textwidth]{res7}} \\
  $[256\times256]$ & $[512\times512]$ \\
\end{tabular}
\normalsize
\end{frame}

%------------------------------------------------
\begin{frame}
\frametitle{Время работы}
\small
\begin{tikzpicture}
\begin{axis}[
    width = 180pt,
    xlabel = {N},
    ylabel = {Время (сек)},
    /pgf/number format/1000 sep={},
    legend style={
        at={(1.6,0.3)},
        anchor=south
    }
]
\addplot coordinates {
    (128, 1.9) (256, 2.1) (512, 2.5) (1024, 2.7)
};
\addplot coordinates {
    (128, 2.6) (256, 3) (512, 3.3) (1024, 3.5)
};
\addplot coordinates {
    (128, 4.1) (256, 5) (512, 5.4) (1024, 5.6)
};
\legend{$k_0 = 416$, $k_0 = 581$, $k_0 = 736$}
\end{axis}
\end{tikzpicture}
\normalsize
\begin{block}{Тестовое окружение}
g++ 4.9.2, Debian sid(unstable)\\
Intel(R) Core(TM) i7-4770 CPU @ 3.40GHz
\end{block}
\end{frame}
%------------------------------------------------

%------------------------------------------------
\begin{frame}
\frametitle{Сравнение}
\center
\small
\begin{tabular}{ p{70pt}cp{70pt}p{70pt} }
  Исходное \mbox{изображение} & Размер & Время (моя реализация) & Время (1-D Python)\\
  \fbox{\includegraphics[width=0.08\textwidth]{s13tiny}} & $[64\times64]$ & 0.6 сек & 1 мин 50 сек\\
  \fbox{\includegraphics[width=0.08\textwidth]{s13smaller}} & $[128\times128]$ & 3 сек & 28 мин 20 сек\\
  \fbox{\includegraphics[width=0.08\textwidth]{s20smaller}} & $[128\times128]$ & 4.9 сек & 14 мин 13 сек\\
  \fbox{\includegraphics[width=0.08\textwidth]{s_1_3_smaller}} & $[128\times128]$ & 2.3 сек & 42 мин 6 сек\\
  \fbox{\includegraphics[width=0.08\textwidth]{s_1_2}} & $[1024\times1024]$ & 1 мин 49 сек & ? \\
\end{tabular}
\normalsize
\end{frame}

%------------------------------------------------
\section{Результаты}
%------------------------------------------------
\begin{frame}
\frametitle{Результаты}
\begin{block}{}
\begin{itemize}
  \item Разработан алгоритм, модифицирующий преобразовние Хафа для поиска эллипсов
  \item Полученный алгоритм реализован и протестирован на синтетических и реальных изображениях.
\end{itemize}
\end{block}
\end{frame}

\end{document}
